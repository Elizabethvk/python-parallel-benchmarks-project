% !TeX root = main.tex

\documentclass[a4paper, oneside]{article}
% \usepackage[margin=2cm,bottom=4cm]{geometry}
\addtolength{\oddsidemargin}{-.25in}
\addtolength{\evensidemargin}{-.25in}
\addtolength{\textwidth}{1.in}

\usepackage{style}
\addbibresource{ref.bib}

\title{Benchmarking Python Parallelization: Joblib, Multiprocessing, Concurrent Futures, and Numba}
\date{\today}

\author{Elizabeth Velikova}
\newcommand{\univname}{Sofia University "St. Kliment Ohridski"\\Faculty of Mathematics and Informatics}

% FOR CODE
% Default fixed font does not support bold face
\DeclareFixedFont{\ttb}{T1}{txtt}{bx}{n}{12} % for bold
\DeclareFixedFont{\ttm}{T1}{txtt}{m}{n}{12}  % for normal

% Custom colors
\usepackage{color}
\definecolor{deepblue}{rgb}{0,0,0.5}
\definecolor{deepred}{rgb}{0.6,0,0}
\definecolor{deepgreen}{rgb}{0,0.5,0}

\newcommand\pythonstyle{\lstset{
language=Python,
basicstyle=\ttm,
morekeywords={self, stdev, assert},% Add keywords here
keywordstyle=\ttb\color{deepblue},
emph={MyClass,__init__},          % Custom highlighting
emphstyle=\ttb\color{deepred},    % Custom highlighting style
stringstyle=\color{deepgreen},
frame=tb,                         % Any extra options here
showstringspaces=false
}}

% Python environment
\lstnewenvironment{python}[1][]
{
\pythonstyle
\lstset{#1}
}
{}

% Python for external files
\newcommand\pythonexternal[2][]{{
\pythonstyle
\lstinputlisting[#1]{#2}}}

% Python for inline
\newcommand\pythoninline[1]{{\pythonstyle\lstinline!#1!}}

% Title stuff
\newcommand{\shorttitle}{Benchmarking Python Parallelization}

\begin{document}
\begin{titlepage}
    \begin{center}
        \vspace*{-2.3cm}
        \includegraphics[height=3cm]{resources/su_logo.png}

        \vspace*{.06\textheight}
        {\scshape\large \univname\par}\vspace{2.5cm}

        {\huge \bfseries{\thetitle}\par}\vspace{0.7cm}
        \textsc{\small in }\\[0.6cm]
        \textsc{\Large Parallel Programming}\\[0.5cm]\vspace{0.5cm}
        \textsc{\normalsize School year 2024/25}\\[0.5cm]\vspace{2cm}


        \begin{minipage}[t]{0.4\textwidth}
            \begin{flushleft} \large
                \emph{Supervisor:}\\[0.7cm]
                Prof. Krassen Stefanov\\[0.7cm]
                Nikolay Shegunov\\[0.4cm]
            \end{flushleft}
        \end{minipage}
        \begin{minipage}[t]{0.4\textwidth}
            \begin{flushright} \large
                \emph{Student:}\\[0.7cm]
                Elizabeth Velikova Koleva \\[0.5cm]
                University ID Number: 4MI3400657 \\[0.4cm]
            \end{flushright}
        \end{minipage}

        \vspace*{.11\textheight}

        Sofia\\[0.2cm]
        {31.01.2025}

    \end{center}
\end{titlepage}
\tableofcontents
\listoffigures
\listoftables
\newpage
\section{Task Objective}
The goal of this project is to analyze and compare different parallelization techniques in Python to determine the most efficient method for executing CPU-bound tasks. 

This is done by implementing and benchmarking four popular parallelization approaches:

\begin{enumerate}
    \item Joblib (Process-based parallel execution)
    \item Multiprocessing (Using Python’s multiprocessing module)
    \item Concurrent Futures (High-level API for process/thread pools)
    \item Numba (Just-In-Time compilation for optimized execution)
\end{enumerate}

In order to be able to accurately comment on the differences between approaches, the number of iterations will be slightly altered with a factor of 100, bringing the total number of samples to 100 000 000.

The results will help identify the best approach for maximizing computational efficiency in Python programs.


TODO:
ThreadPool for IO bound jobs - not our case
Pool for CPU bound jobs - our case

If have time - ThreadPoolExecutor and ProcessPoolExecutor
% https://docs.python.org/3.6/library/concurrent.futures.html?highlight=concurrent%20futures#threadpoolexecutor
% https://docs.python.org/3.6/library/concurrent.futures.html?highlight=concurrent%20futures#processpoolexecutor


\section{Problem Statement}

Many computational problems involve processing large amounts of data or performing heavy mathematical calculations. In Python, parallelization can significantly speed up these operations by utilizing multiple CPU cores. However, choosing the right parallelization method depends on multiple factors, including:

\begin{itemize}
    \item Python’s Global Interpreter Lock (GIL) – Affects thread-based parallelism.
    \item Process vs. Thread Overhead – Processes avoid the GIL but introduce inter-process communication (IPC) overhead.
    \item Computational vs. I/O-bound Tasks – Some methods work better for CPU-heavy operations, while others are optimized for I/O tasks.
\end{itemize}

This project implements a CPU-intensive operation (computing the sum of squares of numbers) using different parallelization techniques and compares their execution times.

\section{Algorithms Used}
\subsection{Benchmark isolation}
In order to isolate the benchmark code from the disturbances of the host system an precaution has been taken - the code is being run in a container. This will further exaggerate the differences between each of the techniques and algorithms.

\subsection{Reproducibility of results}
Using pseudorandom distribution with the default seed in Python (which is the current time in seconds since Epoch) leads to unreliable reproducibility on different machines. The approach to mitigate this issue is that we provide an external seed that is chosen as some integer (by default 3735928559).

\subsection{Sequential}
The code can be found at \figref{fig:sequential}

\section{Results Description}
\section{Future Improvements}
\subsection{Sequential}
The sequential approach would benefit from using vectorised instructions in the fashion on SIMD.

Moreover, optimizing the linear case will benefit all other approaches. Inherently all other techniques deconstruct into some number of sequential calculations.
\section{Conclusion}

\appendix
\section{Code Fragments}
\begin{figure}[h!]
    \centering
    \pythonexternal{./resources/code/sequential.py}
    \caption{Sequential stochastic algorithm}
    \label{fig:sequential}
\end{figure}

\section{Timing Results}
% \begin{table}[h!]
%     \centering
%     \begin{tabular}{c | c}
%     \end{tabular}
%     \caption{Results from sequential calculation}
% \end{table}

% \printbibliography
\end{document}